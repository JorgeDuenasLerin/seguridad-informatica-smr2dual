%%%%%%%%%%%%%%%%%%%%%%%%%%%%%%%%%%%%%%%%%
% Programming/Coding Assignment
% LaTeX Template
%
% This template has been downloaded from:
% http://www.latextemplates.com
%
% Original author:
% Ted Pavlic (http://www.tedpavlic.com)
%
% Note:
% The \lipsum[#] commands throughout this template generate dummy text
% to fill the template out. These commands should all be removed when 
% writing assignment content.
%
% This template uses a Perl script as an example snippet of code, most other
% languages are also usable. Configure them in the "CODE INCLUSION 
% CONFIGURATION" section.
%
%%%%%%%%%%%%%%%%%%%%%%%%%%%%%%%%%%%%%%%%%

%----------------------------------------------------------------------------------------
%	PACKAGES AND OTHER DOCUMENT CONFIGURATIONS
%----------------------------------------------------------------------------------------

\documentclass{article}
\usepackage[utf8]{inputenc}
\usepackage{listingsutf8}
\usepackage[spanish]{babel}




\usepackage{fancyhdr} % Required for custom headers
\usepackage{lastpage} % Required to determine the last page for the footer
\usepackage{extramarks} % Required for headers and footers
\usepackage[usenames,dvipsnames]{color} % Required for custom colors
\usepackage{graphicx} % Required to insert images
\usepackage{listings} % Required for insertion of code
\usepackage{courier} % Required for the courier font
\usepackage{lipsum} % Used for inserting dummy 'Lorem ipsum' text into the template
\usepackage{svg}
\usepackage{attachfile}
\usepackage{currfile}
\usepackage{multicol}
\usepackage{alltt}
\usepackage{framed}

\hypersetup{
colorlinks,
citecolor=black,
filecolor=black,
linkcolor=black,
urlcolor=blue
}


% Margins
\topmargin=-0.45in
\evensidemargin=0in
\oddsidemargin=0in
\textwidth=6.5in
\textheight=9.0in
\headsep=0.25in

\linespread{1.1} % Line spacing

% Set up the header and footer
\pagestyle{fancy}
%\lhead{\hmwkAuthorName} % Top left header
\lhead{\hmwkClass}
\rhead{\hmwkTitle} % Top right header
\chead{} % Top center head
\lfoot{\lastxmark} % Bottom left footer
\cfoot{} % Bottom center footer
\rfoot{ \thepage\ / \protect\pageref{LastPage}} % Bottom right footer
\renewcommand\headrulewidth{0.4pt} % Size of the header rule
\renewcommand\footrulewidth{0.4pt} % Size of the footer rule

\setlength\parindent{0pt} % Removes all indentation from paragraphs

%----------------------------------------------------------------------------------------
%	CODE INCLUSION CONFIGURATION
%----------------------------------------------------------------------------------------

\definecolor{MyDarkGreen}{rgb}{0.0,0.4,0.0} % This is the color used for comments
\lstloadlanguages{Perl} % Load Perl syntax for listings, for a list of other languages supported see: ftp://ftp.tex.ac.uk/tex-archive/macros/latex/contrib/listings/listings.pdf
\lstset{language=Perl, % Use Perl in this example
frame=single, % Single frame around code
basicstyle=\small\ttfamily, % Use small true type font
keywordstyle=[1]\color{Blue}\bf, % Perl functions bold and blue
keywordstyle=[2]\color{Purple}, % Perl function arguments purple
keywordstyle=[3]\color{Blue}\underbar, % Custom functions underlined and blue
identifierstyle=, % Nothing special about identifiers                                         
commentstyle=\usefont{T1}{pcr}{m}{sl}\color{MyDarkGreen}\small, % Comments small dark green courier font
stringstyle=\color{Purple}, % Strings are purple
showstringspaces=false, % Don't put marks in string spaces
tabsize=5, % 5 spaces per tab
%
% Put standard Perl functions not included in the default language here
morekeywords={rand},
%
% Put Perl function parameters here
morekeywords=[2]{on, off, interp},
%
% Put user defined functions here
morekeywords=[3]{test},
%
morecomment=[l][\color{Blue}]{...}, % Line continuation (...) like blue comment
numbers=left, % Line numbers on left
firstnumber=1, % Line numbers start with line 1
numberstyle=\tiny\color{Blue}, % Line numbers are blue and small
stepnumber=5 % Line numbers go in steps of 5
}

% Creates a new command to include a perl script, the first parameter is the filename of the script (without .pl), the second parameter is the caption
\newcommand{\perlscript}[2]{
\begin{itemize}
\item[]\lstinputlisting[caption=#2,label=#1]{#1.pl}
\end{itemize}
}

%----------------------------------------------------------------------------------------
%	DOCUMENT STRUCTURE COMMANDS
%	Skip this unless you know what you're doing
%----------------------------------------------------------------------------------------

% Header and footer for when a page split occurs within a problem environment
\newcommand{\enterProblemHeader}[1]{
%\nobreak\extramarks{#1}{#1 continued on next page\ldots}\nobreak
%\nobreak\extramarks{#1 (continued)}{#1 continued on next page\ldots}\nobreak
}

% Header and footer for when a page split occurs between problem environments
\newcommand{\exitProblemHeader}[1]{
%\nobreak\extramarks{#1 (continued)}{#1 continued on next page\ldots}\nobreak
%\nobreak\extramarks{#1}{}\nobreak
}

\setcounter{secnumdepth}{0} % Removes default section numbers
\newcounter{homeworkProblemCounter} % Creates a counter to keep track of the number of problems

\newcommand{\homeworkProblemName}{}
\newenvironment{homeworkProblem}[1][]{ % Makes a new environment called homeworkProblem which takes 1 argument (custom name) but the default is "Problem #"
\stepcounter{homeworkProblemCounter} % Increase counter for number of problems
\renewcommand{\homeworkProblemName}{Ejercicio \arabic{homeworkProblemCounter} #1} % Assign \homeworkProblemName the name of the problem
\section{\homeworkProblemName} % Make a section in the document with the custom problem count
\enterProblemHeader{\homeworkProblemName} % Header and footer within the environment
}{
\exitProblemHeader{\homeworkProblemName} % Header and footer after the environment
%\clearpage
}


\newcommand{\problemAnswer}[1]{ % Defines the problem answer command with the content as the only argument
\noindent\framebox[\columnwidth][c]{\begin{minipage}{0.98\columnwidth}#1\end{minipage}} % Makes the box around the problem answer and puts the content inside
}

\newcommand{\homeworkSectionName}{}
\newenvironment{homeworkSection}[1]{ % New environment for sections within homework problems, takes 1 argument - the name of the section
\renewcommand{\homeworkSectionName}{#1} % Assign \homeworkSectionName to the name of the section from the environment argument
\subsection{\homeworkSectionName} % Make a subsection with the custom name of the subsection
\enterProblemHeader{\homeworkProblemName\ [\homeworkSectionName]} % Header and footer within the environment
}{
\enterProblemHeader{\homeworkProblemName} % Header and footer after the environment
}

%----------------------------------------------------------------------------------------
%	NAME AND CLASS SECTION
%----------------------------------------------------------------------------------------

\newcommand{\hmwkTitle}{Redefine el hmwkTitle} % Assignment title
\newcommand{\hmwkDueDate}{asdfadsf} % Due date
\newcommand{\hmwkClass}{Seguridad informática} % Course/class
\newcommand{\hmwkClassTime}{} % Class/lecture time
\newcommand{\hmwkClassInstructor}{} % Teacher/lecturer
\newcommand{\hmwkAuthorName}{Álvaro González Sotillo} % Your name

%----------------------------------------------------------------------------------------
%	TITLE PAGE
%----------------------------------------------------------------------------------------

\title{
\vspace{2in}
\textmd{\textbf{\hmwkClass:\ \hmwkTitle}}\\
\vspace{0.1in}\large{\textit{\hmwkClassInstructor\ \hmwkClassTime}}
\vspace{3in}
}

\author{\textbf{\hmwkAuthorName}}
\date{} % Insert date here if you want it to appear below your name


%----------------------------------------------------------------------------------------

\usepackage{fancybox}
\newcommand{\codigo}[1]{\texttt{#1}}


% CUADRITO
\newsavebox{\fmboxx}
\newenvironment{cuadrito}[1][14cm]
{\noindent \begin{center} \begin{lrbox}{\fmboxx}\noindent\begin{minipage}{#1}}
{\end{minipage}\end{lrbox}\noindent\shadowbox{\usebox{\fmboxx}} \end{center}}





\newenvironment{entradasalida}[2][14cm]
{
  \newcommand{\elnombredelafiguradeentradasalida}{#2}
  \begin{figure}[h]
    \begin{cuadrito}[#1]
      \begin{scriptsize}
\begin{alltt}
}
{%
\end{alltt}%
      \end{scriptsize}%
    \end{cuadrito}%
    \caption{\elnombredelafiguradeentradasalida}
  \end{figure}
}


\newenvironment{entradasalidacols}[2][14cm]
{
\newcommand{\elnombredelafiguradeentradasalida}{#2}
%\begin{wrapfigure}{}{0.1}
\begin{cuadrito}[#1]
\begin{scriptsize}
\begin{alltt}
}
{
\end{alltt}
\end{scriptsize}
\end{cuadrito}
\captionof{figure}{\elnombredelafiguradeentradasalida}
%\end{wrapfigure}
}


\newcommand{\ficheroautoref}[0]{

\attachfile[mimetype=text/plain,
description={La plantilla},
subject={La plantilla}]
{../common/plantilla-ejercicio.tex}

\attachfile[mimetype=text/plain,
description={El fichero TEX original para crear este documento, no sea que se nos pierda},
subject={El fichero TEX original para crear este documento, no sea que se nos pierda}]
{\currfilename}}

\newcommand{\entradausuario}[1]{\textit{\textbf{#1}}}

\newcommand{\enlace}[2]{\textcolor{blue}{{\href{#1}{#2}}}}

\newcommand{\adjuntarfichero}[3]{
\textattachfile[mimetype=text/plain,
color={0 0 0},
description={#3},
subject={#1}]
{#1}
{\textcolor{blue}{\codigo{#2}}}
}

\newcommand{\adjuntardoc}[2]{
\textattachfile[mimetype=text/plain,
color={0 0 0},
description={#1},
subject={#1}]
{#1}
{\textcolor{blue}{#2}}
}


\newcommand{\plantilladeclase}[2]{
\adjuntarfichero{#1.java}{#2}{Plantilla para la clase #2}
}

\lstset{
  literate=%
    {á}{{\'a}}1
    {é}{{\'e}}1
    {í}{{\'i}}1
    {ó}{{\'o}}1
    {ú}{{\'u}}1
    {Á}{{\'A}}1
    {É}{{\'E}}1
    {Í}{{\'Y}}1
    {Ó}{{\'O}}1
    {Ú}{{\'U}}1
}

\renewcommand{\lstlistingname}{Listado}
\captionsetup[lstlisting]{font={footnotesize},labelfont=bf,position=bottom}
\captionsetup[figure]{font={footnotesize},labelfont=bf}
\lstnewenvironment{listadojava}[1]
{
  \lstset{captionpos=b,caption={#1},language=Java,frame=single,basicstyle=\footnotesize\ttfamily,showstringspaces=false,numbers=left,xleftmargin=2em}
}
{
}

% LISTADO SHELL
\lstnewenvironment{listadoshell}[2][]
{
  \lstset{captionpos=b,caption={#2},label={#1},language=bash,frame=single,basicstyle=\scriptsize\ttfamily,showstringspaces=false,numbers=none}
}
{
}

% LISTADO TXT
\lstnewenvironment{listadotxt}[2][]
{
  \lstset{inputencoding=utf8,captionpos=b,caption={#2},label={#1},frame=single,basicstyle=\scriptsize\ttfamily,showstringspaces=false,numbers=none}
}
{
}


\newcommand{\graficosvg}[3][14cm]{
\begin{figure}[htbp]
\centering
\textattachfile{#2.svg}{
\color{black}
\includesvg[width=#1]{#2}
}
\caption{#3}
\end{figure}
}


\newcommand{\graficosvguml}[3][7cm]{
  \texttt{\graficosvg[#1]{#2}{#3}}
}


\newcommand{\primerapagina}{
\newpage
\tableofcontents
{\ficheroautoref}
\newpage
}

\usepackage{eurosym}

\graphicspath{{./},{09/}}



\newcommand{\nombreprogramaeventos}{Gestieven}
\newcommand{\nombreempresa}{Kronos}
\renewcommand{\hmwkTitle}{Análisis de riesgos \nombreempresa}


\usepackage{blindtext}

\begin{document}

% \maketitle

% ----------------------------------------------------------------------------------------
%	TABLE OF CONTENTS
% ----------------------------------------------------------------------------------------

% \setcounter{tocdepth}{1} % Uncomment this line if you don't want subsections listed in the ToC

\primerapagina

\setlength{\parindent}{2em}
\setlength{\parskip}{1em}


\section{Enunciado}

La empresa {\nombreempresa} S.A. Se dedica a la organización de eventos. Por ejemplo, organizan fiestas de presentación de productos, desfiles de moda, joyería, grandes actos públicos, como festivales musicales, representaciones teatrales en fiestas de grandes ciudades, etc.

En su sede central, situada en una céntrica calle de Madrid, diponen de unas instalaciones con varios despachos. En la oficina trabajan varios representantes, cuya función principal es estar en contacto con proveedores y clientes, para terminar de organizar el evento.

Cada evento es único, así que el trabajo humano es fundamental, pero las nuevas tecnologías ayudan a estar en contacto.

Cada representante puede conectarse a la red wifi de la empresa con su portatil. En esa red, que es privada, existe una aplicación (llamada {\nombreprogramaeventos}) en la que se almacenan los datos de clientes, eventos y proveedores. A dicha aplicación se accede con un navegador (es decir, es una Aplicación web), pero el servidor es privado.
Para acceder a dicha aplicación, es necesario estar conectado a la wifi de la oficina (no se accede desde Internet) y además introducir un nombre de usuario y una contraseña. Un representante tiene acceso a los datos de sus eventos y de sus clientes, a los datos de todos los proveedores, y a la información de eventos ya realizados y los clientes que los encargaron, así como a los datos del representante que gestionó el evento.

Para mayor comodidad de los representantes, la red wifi permite también el acceso a internet... pero supuestamente, desde Internet no se puede acceder a la aplicación de {\nombreprogramaeventos}.

Un día, un comercial estaba tranquilamente trabajando con su portátil... había iniciado sesión en {\nombreprogramaeventos}, y estaba revisando algunos de los datos a los que estaba autorizado a ver. De repente, se encendió la luz de la webcam. Ésta se puso en marcha. También observó que aparecía en la barra de windows, durante un par de segundos el nombre de un programa que desconocía.

Pasó el tiempo y el incidente fue tomado simplemente como una anécdota sospechosa. Pero, al cabo de un par de semanas, un cliente llamó a nuestro representante y le informó de que cancelaba un evento importante, cuyos detalles eran supuestamente secretos porque se había puesto en contacto con él otro representante de otra empresa ofreciéndole la realización del evento a menor coste. Este nuevo representante conocía detalles que sólo en {\nombreempresa} S.A. debían saber.

El proyecto cancelado iba a ser facturado por unos 720000{\euro}. Después de pagar a proveedores y después de impuestos, el beneficio neto para la empresa sería de unos 65000{\euro}, de los cuales, el 10\% supondría una jugosa comisión para el representante.

El cliente no cedió el proyecto a la otra empresa de eventos, sino que lo canceló por completo: si poca confianza le inspiraba {\nombreempresa}, menos confianza le inspiraba otra empresa que intentaba llevarse el contrato del evento.

El incidente alarmó a la gerencia de la empresa, que nos encarga que estudiemos el caso, para que no vuelva a ocurrir que alguien de la competencia intenta llevarse un evento... y no sólo eso, sino evitar la falta de credibilidad que va a tener durante un tiempo ésta empresa.

\subsection{Parte 1: Análisis de riesgos}


\begin{itemize}
\item Realiza un análisis de riesgos de éste caso, siguiendo el esquema que conoces del análisis.
\item Plásmalo en un documento hecho con tu procesador de textos favorito. Puedes utilizar la \adjuntardoc{SI-T-01-Plantilla-analisis-riesgos.docx}{plantilla adjunta}
  \begin{itemize}
  \item Disponible también en
    \enlace{https://alvarogonzalezsotillo.github.io/seguridad-informatica-smr2dual/apuntes/1/SI-T-01-Plantilla-analisis-riesgos.docx}{este enlace}
\end{itemize}

\item Recuerda que en un análisis de riesgos:
  \begin{itemize}
  \item Identificamos elementos relacionados con el ataque (si se hubiera producido, o con el supuesto ataque en    caso de un temor)
  \item Amenazas, riesgos, activos implicados, vulnerabilidades de cada activo a las amenazas
  \item Describimos el impacto del ataque (si se hubiera producido) o de las posibles amanezas en caso de que se materializasen
  \item Describimos las actividades de negocio afectadas y en qué medida
  \item Describimos el objetivo de quien nos encarga el análisis (¿qué queremos lograr?)
  \item Enumeramos las medidas de seguridad que existen y evaluamos su eficacia
  \item Proponemos soluciones (medidas de seguridad), de manera ordenada, con descripción de las bondades de ésta, sus costes (iniciales y de mantenimiento).
  \item Sugerimos el conjunto de medidas de seguridad que pensamos más adecuado adecuado de entre las aportadas.
  \end{itemize}
\item Recuerda que un análisis de riesgos es un documento técnico, con el estudio de un caso por parte de un profesional: tú, dirigido a una o varias personas implicadas en la gerencia de una empresa u organización.
\item Aun no estamos en codiciones de proponer las cosas con mucho detalle técnico, pero nadie va a cuestionar tus aportes en materia de medidas de seguridad. Tan sólo, que el análisis sea correcto en sus formas.
\end{itemize}

\subsection{Parte 2: Políticas de seguridad.}


\begin{itemize}
\item ¿Hay alguna norma de seguridad que quieras que los empleados de {\nombreempresa} cumplan? Si es así, detállala en  el cuadro de notas que hay al final de éste ejercicio.
  
\item Con una o dos líneas por cada cosa que sugieras es   suficiente...
  
\item Si no es necesario tener en cuenta ninguna consideración adicional que deba ser conocida por el personal  de {\nombreempresa}, indícalo también, mencionando que no es necesario incluir ninguna política de seguridad nueva.
\end{itemize}

Recuerda que son normas dirigidas a personal no técnico: lenguaje sencillo y claro. La gerencia de la empresa lo leerá y si aprueba nuestras sugerencias, se incluirán en el documento de Políticas de Seguridad de la empresa.

\section{Qué se valorará}
\begin{itemize}
\item Que incluyas todos los apartados del análisis de riesgos, sin salirse demasiado del caso expuesto.
\item Que lo que pongas en cada apartado, pertenezca, en efecto a él
\item Que esté correctamente redactado como para que nuestro lector lo entienda:
  \begin{itemize}
  \item El lector de un análisis de riesgos es alguien de la gerencia de la empresa, que debe decidir sobre gastar  dinero en medidas de seguridad.
  \end{itemize}
\item La apariencia profesional del análisis:
  \begin{itemize}
  \item Estética
  \item Organización
  \item Homogeneidad de formatos y estilos
  \end{itemize}
\end{itemize}

\section{Instrucciones de entrega}
\begin{itemize}
\item El ejercicio se realizará y entregará de manera individual.
  \begin{itemize}
  \item Solo se admiten trabajos en pareja, si en clase es necesario compartir ordenador.
  \end{itemize}
\item Entrega tu trabajo en formato \textbf{doc}, \textbf{docx}, \textbf{odt} o \textbf{pdf}.
\item Sube el documento a \enlace{http://aulavirtual2.educa.madrid.org/course/category.php?id=2724}{la tarea correspondiente en el aula virtual}
\item Presta atención al plazo de entrega (con fecha y hora).
  
\end{itemize}
\end{document}
